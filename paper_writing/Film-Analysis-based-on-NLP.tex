\documentclass[12pt]{article}
\usepackage{ctex} % 中文的宏包
\usepackage{booktabs} % For \toprule, \midrule and \bottomrule
\usepackage{siunitx} % Formats the units and values
\usepackage{pgfplotstable} % Generates table from .csv
\usepackage{csvsimple}
\usepackage{cite}
\usepackage{indentfirst}
\usepackage{graphicx} % 插入圖片的宏包
\usepackage{float} % 設置圖片浮動位置的宏包
\usepackage{subfigure} % 插入多圖時用子圖顯示宏包
\usepackage{listings} % 代碼塊宏包
\usepackage{color} % 代碼高亮
%\usepackage[colorlinks,linkcolor=blue]{hyperref} % URL 包
\usepackage[pdf]{graphviz}
\usepackage{alphalph}
% \usetheme{Madrid}
\renewcommand*{\thesubfigure}{(\arabic{subfigure})}

\definecolor{dkgreen}{rgb}{0,0.6,0}
\definecolor{gray}{rgb}{0.5,0.5,0.5}
\definecolor{mauve}{rgb}{0.58,0,0.82}
\bibliographystyle{unsrt} 
\lstset{ %
    %language=Octave,                % the language of the code
    basicstyle=\scriptsize\Hack,           % the size of the fonts that are used for the code
    numbers=none,                   % where to put the line-numbers
    numberstyle=\tiny\color{gray},  % the style that is used for the line-numbers
    stepnumber=2,                   % the step between two line-numbers. If it's 1, each line 
                                    % will be numbered
    numbersep=3pt,                  % how far the line-numbers are from the code
    backgroundcolor=\color{white},      % choose the background color. You must add \usepackage{color}
    showspaces=false,               % show spaces adding particular underscores
    showstringspaces=false,         % underline spaces within strings
    showtabs=false,                 % show tabs within strings adding particular underscores
    frame=single,                   % adds a frame around the code
    rulecolor=\color{black},        % if not set, the frame-color may be changed on line-breaks within not-black text (e.g. commens (green here))
    tabsize=2,                      % sets default tabsize to 2 spaces
    captionpos=b,                   % sets the caption-position to bottom
    breaklines=true,                % sets automatic line breaking
    breakatwhitespace=false,        % sets if automatic breaks should only happen at whitespace
    title=\lstname,                   % show the filename of files included with \lstinputlisting;
                                    % also try caption instead of title
    keywordstyle=\color{blue},          % keyword style
    commentstyle=\color{dkgreen},       % comment style
    stringstyle=\color{mauve},         % string literal style
    escapeinside={\%*}{*},            % if you want to add LaTeX within your code
    morekeywords={*,...}               % if you want to add more keywords to the set
}
\setCJKmainfont{黑体} % 主要字體 Noto Serif
\setmainfont{Times New Roman}
\newfontfamily\Hack{Consolas} % 代碼字體

\title{Film Analysis Based on NLP and Spider}
\author{向兴旺}

\begin{document}
\maketitle
\section{Abstract}

\paragraph{摘\quad 要}
2022, 中国电影市场处于疫情的特殊关头。本文将通过爬虫获取豆瓣上最热门的十部电影的基本信息、
粉丝分布、评分数据,评论数据,并对这些数据利用python进行数据处理和分析,特别的,
对于评论数据,本文将使用NLP技术对其进行分词和情感分析。最终,得到每一部电影的正负面
评价比率、粉丝分布、票房属性,对电影作出分析和推荐报告,更进一步的,我们试图通过这些数据,
分析疫情下的中国电影市场。
\paragraph{关键词}
爬虫 \, NLP \, 电影 \, 分词 \, 情感分析 \,  数据处理 \, 电影市场 \, 票房属性

\section{Film Recomending Module}
当前主要的电影分析手段,仍然是人工分析,而传统人工分析:
\par
1.如果对于影片的评论作完善分析,则逐条逐字查看太过消耗时间。
\par
2.如果仅仅对于影片的评分进行分析,则难以避免水军刷评论的现象。
\par
针对这一情形,构建一个自动化电影分析框架是有其必要价值的。本文主要利用了两种技术:
爬虫,NLP。均采用python实现。
\\
\par
在数据获取,即爬虫阶段,本文主要将其分为两个步骤: 1)http交互:即通过程序,向远程服务器发送web
请求,并接受respose报文,实现与web服务器的交互。主要使用requests module 实现
2)数据处理和保存。即对于接受到的html报文进行解析,提取关键数据,并进行格式化的存储,
便于数据分析时使用。主要使用BeautifulSoup module 和json module 实现。

我们主要获取获取的信息包括:影片的基本信息,影片的评论内容,影片评论人的地址分布,影片评论的分数分布。
\\
\par
在数据分析,也包含两个步骤: 1)图表绘制,即对于所收集的数据,以各种图表做可视化处理。
主要基于matplotlib module 和seaborn module 实现。 2)NLP分析,主要采用SnowNLP module 和
jieba module 实现。\cite{wangWhichMoreSuitable2009}
\par
首先,对于地址和评分的分布,可以考虑用扇形图直观的描述。
但是,对于评论信息,难以直观的表述和呈现,所以需要使用NLP的技术加以分析。
考虑以下方法,首先,一部电影 $i$ ,对于其每一条评论 $k$, 进行情感分析,得到情感分数 $s_{ik}$,
(分数越大,情感越积极)。然后,对于所有评论,综合考虑,得到情感分数的均值$\overline{s_i}$以及情感分数的分布图。
紧接着,分别通过分词和关键词提取技术,分别获取负面评论和正面评论中出现频率较大的词,过滤无价值的 stop words
(主要基于代码目录下stop\_word.txt)。
得出好评观众和差评观众各自的关注点和影片的关键词。 对评论形成完善的分析模型。
\\
\par 
总体而言。本文完成的工作:
\begin{itemize}
    \item 通过爬虫和NLP技术对于电影进行评价性分析
    \item 完成了一个电影分析的代码框架并将相应代码和数据开源
\end{itemize}

\section{}
本项目的完整代码开源在https://github.com/Du-Mu/film\_recomending

\subsection{Data Collection}

首先,本文以豆瓣电影新片榜(https://movie.douban.com/chart)界面为爬虫的首先访问界面,
这一界面列举了最近热映的新片。对于此界面的html代码进行分析,可以发现电影的基本信息都在class=‘nbg’的元素中
。爬取此页面后,我们可以遍历此界面的nbg元素,获取每一个对应电影的link,进一步,请求访问
对应电影的界面和对应电影评论的界面,后面的这一步骤主要由$get\_basic\_data()$,$film\_reviews\_spider()$实现。
\lstset{language=Python}
\begin{lstlisting}
    film_url = 'https://movie.douban.com/chart'
    film_html = requests.get(url=film_url, headers=headers).text
    film_soup = BeautifulSoup(film_html, 'html.parser')
    num = 0
    for i in film_soup.find_all('a', 'nbg'):
        # 解析每一个电影对应的元素,获取对应的电影专业的link
        rating_data = {}
        addr_data = {}
        basic_data = {}
        num+=1
        print('[-]now are processing film ' + str(num))
        film_name = i.get('title')
        film_info = {}
        film_reviews_spider(film_link=i.get('href'), film_info=film_info, rating_data=rating_data, addr_data=addr_data)
        basic_data = get_basic_data(i.get('href'))
\end{lstlisting}

在$get\_basic\_data()$中,针对性的对电影主界面进行分析,在script中type为'application/ld+json'中
发现电影信息的json格式字符串。于是获取对应内容,再将之存储在名为basic\_iofo的dict中。

\lstset{language=Python}
\begin{lstlisting}
def get_basic_data(film_url):
    time.sleep(1)
    film_main_html = requests.get(url=film_url, headers=headers).text
    
    film_main_soup = BeautifulSoup(film_main_html, 'html.parser')

    basic_info = film_main_soup.find('script', type='application/ld+json').text
    
    return json.loads(basic_info, strict = False)
\end{lstlisting}

在film\_reviews\_spider()中,由于一个界面,不能显示过多的评论,于是迭代获取对应界面,
同时,为了防止触发反爬虫措施,我们通过$sleep()$ 降低请求的频率。
在获取到评论界面后,将获取到的评论,分条目存入dict中,同时统计评分和地理位置信息。


\lstset{language=Python}
\begin{lstlisting}
def film_reviews_spider(film_link, film_info, rating_data, addr_data):
    start_num = 0;
    count = 0
    while (start_num < 1000):
        # 迭代获取评论界面
        print('[*]now are loading '+str(start_num)+' comments')
        start_num += 100
        review_url = film_link+'comments?start='+str(start_num)+'&limit=100&status=P&sort=new_score'
        time.sleep(5)
        try:
            review_html = requests.get(url=review_url, headers=headers).text
        except Exception:
            print('[*]fail to parse:'+review_url)
            return False
        review_soup = BeautifulSoup(review_html, 'html.parser')

        for i in review_soup.find_all('div', 'comment'):
            count += 1
            # 找到每一条评论
            addr = i.find('span', 'comment-location').text
            rating = i.contents[1].contents[3].contents[5].get('title')

            comment_info = {
                'id':i.contents[1].contents[3].contents[1].text,
                'content':i.find('span', 'short').text,
                'addr':addr,
                'rating':rating
            }

            if (rating_data.get(rating) == None and len(rating) == 2):
                rating_data[rating] = 1
            elif (len(rating) == 2):
                rating_data[rating] += 1
            # 统计分数信息
            if (addr_data.get(addr) == None):
                addr_data[addr] = 1
            else:
                addr_data[addr] += 1
            # 统计地址信息
            film_info[count] = comment_info
            # 存入评论信息

    return True
\end{lstlisting}

最后,将获取到的数据,分别存放到不同下的json文件中,方便后续分析和读取
\lstset{language=Python}
\begin{lstlisting}
try:
    os.mkdir('./film_data/film'+str(num))
except Exception:
    continue

file = open('./film_data/film'+str(num)+'/comments.json', 'w')
json.dump(film_info, fp=file, sort_keys=True, separators=(',', ': '), indent=4, ensure_ascii=False)
file.close()

file_addr = open('./film_data/film'+str(num)+'/addr.json', 'w')
json.dump(addr_data, fp=file_addr, sort_keys=True, separators=(',', ': '), indent=4, ensure_ascii=False)
file_addr.close()

file_rating = open('./film_data/film'+str(num)+'/rating.json', 'w')
json.dump(rating_data, fp=file_rating, sort_keys=True, separators=(',', ': '), indent=4, ensure_ascii=False)
file_rating.close()

file_basic = open('./film_data/film'+str(num)+'/basic.json', 'w')
json.dump(basic_data, fp=file_basic, sort_keys=True, separators=(',', ': '), indent=4, ensure_ascii=False)
file_basic.close()
\end{lstlisting}

\subsection{Data Analysis}
首先我们对电影基本信息进行分析,将基本各个电影的基本信息以csv表格的形式列出。
\lstset{language=Python}
\begin{lstlisting}
def analysis_basic_info(csv_file_name):
    header = ['No','name','data','genre','data','rating_count', 'rating_value']
    csv_file = open(csv_file_name, 'w')
    writer = csv.writer(csv_file)
    writer.writerow(header)

    for i in range(1, 11):
        file = open('./film_data/film'+str(i)+'/basic.json', 'r')
        basic_info = json.load(fp=file)
        row = [
            i,
            basic_info['name'],
            basic_info['datePublished'],
            basic_info['genre'],
            basic_info['aggregateRating']['ratingCount'],
            basic_info['aggregateRating']['ratingValue'],
        ]
        writer.writerow(row)
        file.close()
    csv_file.close()
\end{lstlisting}

其次,我们处理地理分布信息,根据之前得到的地理分布人数的json文件,得到地理分布人数的dict,
然后,根据分布人数,作出对应分布图。
\lstset{language=Python}
\begin{lstlisting}
def analysis_addr_info(file_name, film_no):
    file = open(file_name, 'r')
    addr_info = json.load(fp=file)
    file.close()
    
    addr_info['else'] = 0
    for key in list(addr_info.keys()):
        if (addr_info[key] < 8):
            addr_info['else'] += addr_info[key]
            del addr_info[key]

    plt.rcParams['font.sans-serif']=['KaiTi','SimHei','FangSong']
    plt.figure(figsize = (8,8))
    explode = []
    for i in range(0, len(addr_info)):
        explode.append(i*0.01)
    plt.pie(addr_info.values(), labels=addr_info.keys(), explode = explode, autopct = '%1.2f%%',
        pctdistance = 0.8, labeldistance = 0.9)
    plt.title('addr_info')
    plt.savefig('./paper_writing/images/addr'+str(film_no)+'.jpg')
\end{lstlisting}

类似的,我们处理分数分布信息:
\lstset{language=Python}
\begin{lstlisting}
def analysis_rating_info(file_name, film_no):
    file = open(file_name, 'r')
    addr_info = json.load(fp=file)
    file.close()
    
    plt.rcParams['font.sans-serif']=['KaiTi','SimHei','FangSong']
    plt.figure(figsize = (8,8))
    explode = []
    for i in range(0, len(addr_info)):
        explode.append(i*0.01)
    plt.pie(addr_info.values(), labels=addr_info.keys(), explode = explode, autopct = '%1.2f%%',
        pctdistance = 0.8, labeldistance = 0.9)
    plt.title('rating_info')
    plt.savefig('./paper_writing/images/rating'+str(film_no)+'.jpg')
\end{lstlisting}


最后,处理评论信息。对于评论信息,主要进行了两项处理,情感分析和关键词提取。

关键词提取:分词和关键词主要基于snownlp和jieba实现的TextRank算法,对于一段文本,构建一段基于各种词与词和词
之间联系的无向图,再根据图的连接,更新每个词的权重。\cite{mihalceaTextRankBringingOrder}


情感分析:对于提取出的关键词,分析其情感因子,再结合其在图中的位置和关键程度,给予其相应的系数,
综合而言,得到整段评论的情感因子。\cite{itoWordLevelContextualSentiment2020}

\lstset{language=Python}
\begin{lstlisting}
def analysis_comments_info(file_name, film_no): 
    stopwords = [line.strip() for line in open('./stop_word.txt', encoding="utf-8").readlines()]
    pos_word = {}
    neg_word = {}

    file = open(file_name, 'r')
    comments_info = json.load(fp=file)
    file.close()
    com_sentiments = []
    for i in range(1, len(comments_info)+1):
        comment = comments_info[str(i)]['content']
        comment = ' '.join(re.findall('[\u4e00-\u9fa5]+', comment, re.S))
        if comment == '':
            continue
        comment_snow = SnowNLP(comment)
        com_sentiments.append(comment_snow.sentiments)
        # 获取情感因子
    sns.distplot(com_sentiments, bins=30)
    plt.savefig('./paper_writing/images/comments'+str(film_no)+'.jpg')

    mean = np.mean(com_sentiments)

    for i in range(1, len(comments_info)+1):
        comment = comments_info[str(i)]['content']
        comment = ' '.join(re.findall('[\u4e00-\u9fa5]+', comment, re.S))
        if comment == '':
            continue
        comment_snow = SnowNLP(comment)
        if comment_snow.sentiments > mean:
            now_word = pos_word
        else:
            now_word = neg_word

        words = jieba.lcut(comment)
        for word in words:
            if word in stopwords:
                continue
            if len(word)<2:
                continue
            if now_word.get(word):
                now_word[word]+=1
            else:
                now_word[word] = 1
            # 过滤无价值词
    pos_item = list(pos_word.items())
    pos_item.sort(key=lambda x: x[1], reverse=True)
    print("pos_word:")
    for i in range(30):
        p_word, word_count = pos_item[i]
        print(p_word+':'+str(word_count))

    neg_item = list(neg_word.items())
    neg_item.sort(key=lambda x: x[1], reverse=True)
    print("neg_word:")
    for i in range(30):
        p_word, word_count = neg_item[i]
        print(p_word+':'+str(word_count))
\end{lstlisting}

\section{Film Analysis}
\subsection{Overview:}

\begin{figure}[h]
    \centering
    \includegraphics[width=15cm]{table.png}
    \caption{电影基本信息统计}
\end{figure}

    
\subsection{西线无战事}
\subsubsection{图表分布}
\begin{figure}[H]
    \centering
    \begin{minipage}[t]{0.48\textwidth}
    \centering
    \includegraphics[width=6cm]{./images/addr1.jpg}
    \caption{观众评论地址分布}
    \end{minipage}
    \begin{minipage}[t]{0.48\textwidth}
    \centering
    \includegraphics[width=6cm]{./images/rating1.jpg}
    \caption{观众评论分数分布}
    \end{minipage}
\end{figure}

\begin{figure}[H]
    \centering
    \includegraphics[width=10cm]{./images/comments1.jpg}
    \caption{影片评论情感分数分布} 
\end{figure}

\subsubsection{Keywords}
正面评论和负面评论关键词
\paragraph{pos\_word:}
战争:295,电影:88,战场:74,反战:70,士兵:60,残酷:53,战争片:51,配乐:43,原著:43,最后:38,死亡:35,一战:33,生命:32,西线:32,故事:31,觉得:31,保罗:30,镜头:30,摄影:28,无战事:28,人类:27,震撼:27,非常:26,和平:26,真的:26,视角:25,停战:25,反思:24,世界:24,虚无:24
\paragraph*{neg\_word:}
战争:9,国家:7,士兵:6,最后:6,棋子:5,戰場:5,地方:5,原著:5,一战:5,原版:5,不會:4,戰爭:4,殘酷:4,真的:4,西线:4,无战事:4,现在:4,有点:4,节奏:4,反战:4,和平:4,导演:3,他們:3,一個:3,生命:3,有人:3,知道:3,震撼:3,主角:3,青年:3

\subsubsection{综合分析}
综合来看,本片整体分数较高,力荐和推荐(好评)占比例72.02\%, 还行(中评)占比22.43\%,
差评占比5.55\%,粉丝主要分布在北京、上海、四川、山东等地区。
正面评价主要关键词包含战争、反战、残酷、配乐、镜头。差评关键词主要包含节奏、原版等。
说明其作为战争电影,在叙事、配乐、镜头上,很好地表现了战争的残酷,
集中体现了反战的价值观,但是情节和原著有一定差别。
\par
总体而言,较为推荐。

\subsection{名侦探柯南\:万圣节的新娘}
\subsubsection{图表分布}
\begin{figure}[H]
    \centering
    \begin{minipage}[t]{0.48\textwidth}
    \centering
    \includegraphics[width=6cm]{./images/addr2.jpg}
    \caption{观众评论地址分布}
    \end{minipage}
    \begin{minipage}[t]{0.48\textwidth}
    \centering
    \includegraphics[width=6cm]{./images/rating2.jpg}
    \caption{观众评论分数分布}
    \end{minipage}
\end{figure}

\begin{figure}[H]
    \centering
    \includegraphics[width=10cm]{./images/comments2.jpg}
    \caption{影片评论情感分数分布} 
\end{figure}
\subsubsection*{Keywords}
正面评论和负面评论关键词
\paragraph*{pos\_word:}
柯南:158,剧场版:143,推理:83,警校:70,剧情:54,动作:54,一部:54,真的:54,五人组:52,最后:49,好看:45,已经:41,故事:40,感觉:36,不错:36,松田:35,这部:34,电影:32,安室:29,确实:28,柯学:28,场面:27,毛利:25,有点:25,情怀:25,部分:25,足球:25,现在:24,炸弹:23,电影院:22
\paragraph*{neg\_word:}
柯南:20,松田:14,真的:14,剧场版:14,推理:13,离谱:11,反派:10,阵平:8,难看:8,安室:8,剧情:8,一部:8,下水道:7,涩谷:7,最后:7,佐藤:6,最佳:6,只能:6,足球:6,几年:6,有点:6,竟然:6,五人组:6,故事:6,警校:6,劇場:6,俄语:5,实在:5,高木:5,涉谷:5

\subsubsection*{综合评价}
综合来看,本片整体水平比较中庸,力荐和推荐(好评)占比41.71\%,
还行(中评)占比46.78\%,差评占比11.08\%,粉丝主要分布在北京、上海、浙江等地。
正面评价主要关键词包括警校、动作、剧情、五人组等,负面评论主要包含推理、离谱、反派、松田、难看等,
说明其作为一部柯南电影,仍然带有很大粉丝向的成分,对于警校五人组的刻画,以及剧情和动作戏,都很好地
满足了粉丝的需求。负面评论主要集中吐槽了反派的塑造和推理仍旧有一些离谱。

总体而言,是一部合格的粉丝向电影。

\subsection*{危笑}
\subsubsection*{图表分布}
\begin{figure}[H]
    \centering
    \begin{minipage}[t]{0.48\textwidth}
    \centering
    \includegraphics[width=6cm]{./images/addr3.jpg}
    \caption{观众评论地址分布}
    \end{minipage}
    \begin{minipage}[t]{0.48\textwidth}
    \centering
    \includegraphics[width=6cm]{./images/rating3.jpg}
    \caption{观众评论分数分布}
    \end{minipage}
\end{figure}

\begin{figure}[H]
    \centering
    \includegraphics[width=10cm]{./images/comments3.jpg}
    \caption{影片评论情感分数分布} 
\end{figure}
\subsubsection*{Keywords}
正面评论和负面评论关键词
\paragraph*{pos\_word:}
恐怖片:79,女主:55,微笑:53,恐怖:53,最后:50,诅咒:50,不错:44,电影:42,故事:40,真的:39,镜头:34,吓人:34,有点:34,结尾:31,剧情:29,怪物:28,创伤:26,结局:24,觉得:23,心理:22,身后:22,氛围:22,音效:22,一惊:22,诡异:21,其实:20,这种:20,惊吓:20,午夜凶铃:20,影片:19
\paragraph*{neg\_word:}
吓人:29,恐怖片:27,恐怖:20,午夜凶铃:18,有点:15,剧情:12,真的:12,最后:12,诅咒:11,女主:11,无聊:11,电影:10,预告片:9,看到:9,身后:8,不会:8,一惊:8,觉得:8,吓死:8,知道:7,镜头:7,这种:7,老套:7,专业:7,微笑:6,设定:6,电影院:6,低级:6,地方:6,前男友:6
\subsubsection*{综合评价}
综合来看,本片整体水平中等偏下,力荐和推荐(好评)占比19.10\%,
中评占比52.98\%, 差评占比27.92\%, 粉丝主要分布在北京、上海、广东,正面评价主要关键词包括恐怖片、女主、微笑、恐怖、诅咒等。
负面关键词主要包括恐怖、午夜凶铃、剧情、老套等。
说明,作为一部恐怖片,对于大多数差评观众而言,有一点惊吓,但是对于一部分恐怖电影
受众来说,有点老套。剧情略有瑕疵。

总体而言,对于想要体验恐怖电影的人来说,可以一看,但是对于其他观众,则不是很推荐。

\subsection*{6$/$45}
\subsubsection*{图表分布}
\begin{figure}[H]
    \centering
    \begin{minipage}[t]{0.48\textwidth}
    \centering
    \includegraphics[width=6cm]{./images/addr4.jpg}
    \caption{观众评论地址分布}
    \end{minipage}
    \begin{minipage}[t]{0.48\textwidth}
    \centering
    \includegraphics[width=6cm]{./images/rating4.jpg}
    \caption{观众评论分数分布}
    \end{minipage}
\end{figure}
\begin{figure}[H]
    \centering
    \includegraphics[width=10cm]{./images/comments4.jpg}
    \caption{影片评论情感分数分布} 
\end{figure}
\subsubsection*{Keywords}
正面评论和负面评论关键词
\paragraph*{pos\_word:}
喜剧:62,哈哈哈:56,搞笑:50,南北:48,电影:39,不错:38,真的:30,统一:29,好笑:29,故事:29,笑点:22,哈哈哈哈:22,共同:20,朝鲜:20,有点:20,结尾:19,剧情:18,彩票:18,韩国:18,和平:17,题材:17,很多:17,喜剧片:16,爆笑:15,一部:15,警备区:15,轻松:15,看到:14,荒诞:14,这种:13
\paragraph*{neg\_word:}
搞笑:27,南北:24,好笑:23,喜剧:14,真的:13,共同:11,警备区:11,有点:11,统一:11,彩票:10,韩国:9,电影:8,无聊:7,女团:7,笑点:6,知道:6,剧情:5,可能:5,好看:5,喜欢:5,野猪:5,故事:4,设定:4,觉得:4,三星:4,真是:4,无厘头:4,地方:4,好久没:4,确实:4
\subsubsection*{综合评价}
综合来看,本片水平中等偏上。好评比例62.20\%,
中评比例31.50\%,差评比例2.30\%,粉丝主要分布于北京、上海,
正面评价主要关键词包括哈哈哈、喜剧、南北、搞笑。
负面评价主要关键词包括搞笑、南北、好笑、彩票、韩国等。
作为一部喜剧电影,完成了其搞笑的使命,同时其中涉及到了一些关于朝鲜和韩国的关系的话题,
有些人可能无所谓,有些人可能较为排斥。

总体而言,如果想看一部能让你放松的喜剧电影,那么这部电影非常推荐。

\subsection*{共助2:国际}
\subsubsection*{图表分布}
\begin{figure}[H]
    \centering
    \begin{minipage}[t]{0.48\textwidth}
    \centering
    \includegraphics[width=6cm]{./images/addr5.jpg}
    \caption{观众评论地址分布}
    \end{minipage}
    \begin{minipage}[t]{0.48\textwidth}
    \centering
    \includegraphics[width=6cm]{./images/rating5.jpg}
    \caption{观众评论分数分布}
    \end{minipage}
\end{figure}
\begin{figure}[H]
    \centering
    \includegraphics[width=10cm]{./images/comments5.jpg}
    \caption{影片评论情感分数分布} 
\end{figure}
\subsubsection*{Keywords}
正面评论和负面评论关键词
\paragraph*{pos\_word:}
玄彬:99,动作:63,电影:60,第一部:53,韩国:51,剧情:42,搞笑:34,喜剧:33,林允儿:30,真的:28,一部:28,好看:27,允儿:25,不错:25,最后:24,共助:23,角色:22,美国:22,爆米花:21,花痴:20,感觉:20,丹尼尔:20,帅哥:20,动作片:19,商业片:19,场面:18,故事:17,看到:16,商业:16,小姨子:16
\paragraph*{neg\_word:}
玄彬:33,韩国:21,电影:13,第一部:11,爆米花:11,剧情:10,允儿:9,真的:9,搞笑:9,丹尼尔:8,动作:8,无聊:8,不能:6,娱乐性:6,反派:6,合格:6,一部:6,好看:5,帅哥:5,南北:5,商业片:5,林允儿:5,有点:5,美国:5,帅气:4,没意思:4,好笑:4,感觉:4,好帅:4,差不多:4
\subsubsection*{综合评价}
总体而言,评价中等,好评率41.43\%,中评占比48.37\%,差评占比10.21\%。
粉丝主要分布于北京。正面评价主要关键词包括玄彬、动作、电影、韩国、剧情、搞笑。
负面评论关键词主要包括玄彬、韩国、电影、爆米花。
是一部一部动作类型的爆米花电影,主演关键词占比高,主要卖点是演员。

总而言之,是一部中规中矩的电影,适合主演的粉丝观看。

\subsection*{黑亚当}
\subsubsection*{图表分布}
\begin{figure}[H]
    \centering
    \begin{minipage}[t]{0.48\textwidth}
    \centering
    \includegraphics[width=6cm]{./images/addr6.jpg}
    \caption{观众评论地址分布}
    \end{minipage}
    \begin{minipage}[t]{0.48\textwidth}
    \centering
    \includegraphics[width=6cm]{./images/rating6.jpg}
    \caption{观众评论分数分布}
    \end{minipage}
\end{figure}
\begin{figure}[H]
    \centering
    \includegraphics[width=10cm]{./images/comments6.jpg}
    \caption{影片评论情感分数分布} 
\end{figure}
\subsubsection*{Keywords}
正面评论和负面评论关键词
\paragraph*{pos\_word:}
电影:129,英雄:103,亚当:102,博士:73,特效:68,剧情:63,角色:60,正义:55,命运:54,超英:52,爆米花:49,强森:45,最后:39,巨石:38,超级:38,漫威:36,故事:33,真的:33,有点:30,协会:29,超人:29,看到:27,已经:26,不错:26,彩蛋:26,这种:25,布鲁斯南:23,一部:23,感觉:22,这部:21
\paragraph*{neg\_word:}
特效:25,剧情:22,电影:19,超英:15,真的:12,漫威:10,这种:9,有点:9,巨石:8,好看:8,最后:7,感觉:7,彩蛋:7,无聊:7,亚当:7,强森:6,反派:6,觉得:6,知道:6,毫无:6,爆米花:6,电影院:5,超级:5,唯一:5,超人:5,意思:5,片子:5,编剧:5,演技:5,鬼子:4
\subsubsection*{综合评价}
整体评价中等偏下,好评占比17.57\%,中评占比49.70\%,差评占比32.73\%,
粉丝主要分布于北京、上海。正面评论中主要关键词为英雄、亚当、博士、特效、剧情、角色,负面评论中主要
关键词为特效、剧情、电影、超英。
是一部典型的超级英雄影片,有着比较充足的特效,但是剧情存在一定问题。

总体而言,不是很推荐观看,仅仅适合想要放松心情的人群。

\subsection*{火山挚恋}
\subsubsection*{图表分布}
\begin{figure}[H]
    \centering
    \begin{minipage}[t]{0.48\textwidth}
    \centering
    \includegraphics[width=6cm]{./images/addr7.jpg}
    \caption{观众评论地址分布}
    \end{minipage}
    \begin{minipage}[t]{0.48\textwidth}
    \centering
    \includegraphics[width=6cm]{./images/rating7.jpg}
    \caption{观众评论分数分布}
    \end{minipage}
\end{figure}
\begin{figure}[H]
    \centering
    \includegraphics[width=10cm]{./images/comments7.jpg}
    \caption{影片评论情感分数分布} 
\end{figure}
\subsubsection*{Keywords}
正面评论和负面评论关键词
\paragraph*{pos\_word:}
火山:329,人类:110,浪漫:92,纪录片:68,影像:59,震撼:47,热爱:46,人生:46,岩浆:43,自然:43,生命:41,素材:41,地球:38,爱情:35,电影:34,真的:34,两人:33,一生:31,渺小:31,极致:31,两个:30,科学家:30,看到:29,剪辑:27,莫里斯:27,一起:26,夫妇:25,危险:25,一种:25,火山爆发:24
\paragraph*{neg\_word:}
火山:16,浪漫:8,北影:6,旁白:5,震撼:4,時間:4,影像:4,极致:3,无聊:3,喜欢:3,故事:3,他們:3,今年:3,觉得:3,四星:2,科学:2,知道:2,野兽:2,完全:2,一点:2,热爱:2,素材:2,纯粹:2,真的:2,配乐:2,两个:2,跳舞:2,活着:2,这部:2,内容:2
\subsubsection*{综合评价}
整体评价非常好,好评率达到了84.03\%,中评占比14.71\%,差评仅占1.26\%,
主要观影人群在北京,好评非常多,在这些好评中,大多都提到了浪漫的关键词,纪录片的手法和影像都给人很大的震撼,
也能给人以关于人生和自然的思考。差评中,主要是部分人会觉得有一些无聊。
一部富有启迪性的纪录片,献给火山爱好者和浪漫主义者的情书,

整体而言,非常推荐,可以给人以很深刻的思考。


\subsection*{福尔摩斯小姐:伦敦厄运}
\subsubsection*{图表分布}
\begin{figure}[H]
    \centering
    \begin{minipage}[t]{0.48\textwidth}
    \centering
    \includegraphics[width=6cm]{./images/addr8.jpg}
    \caption{观众评论地址分布}
    \end{minipage}
    \begin{minipage}[t]{0.48\textwidth}
    \centering
    \includegraphics[width=6cm]{./images/rating8.jpg}
    \caption{观众评论分数分布}
    \end{minipage}
\end{figure}通过 zotero-better-bibtex 插件来设置
\begin{figure}[H]
    \centering
    \includegraphics[width=10cm]{./images/comments8.jpg}
    \caption{影片评论情感分数分布} 
\end{figure} 
\subsubsection*{Keywords}
正面评论和负面评论关键词
\paragraph*{pos\_word:}
福尔摩斯:104,第一部:97,女性:85,推理:76,电影:60,华生:52,好看:51,里亚蒂:49,黑人:46,剧情:43,故事:39,真的:38,最后:37,有点:35,女权:32,喜欢:32,正确:32,感觉:30,可爱:28,很多:27,小姐:27,一部:26,政治:26,部分:25,角色:25,不错:24,女主:24,夏洛克:22,女工:21,系列:19
\paragraph*{neg\_word:}
第一部:20,黑人:16,里亚蒂:14,好看:13,华生:12,无聊:9,真的:7,爆米花:6,一部:6,夏洛克:6,最后:6,女权:5,一点:5,火柴:5,剧情:5,亨利:4,女工:4,有点:4,黑女:4,女性:4,還是:4,很多:4,三星:4,可爱:3,无法:3,接受:3,看看:3,事件:3,小妹:3,星星之火:3
\subsubsection*{综合评价}
总体评价中等, 好评占比30.33\%, 中评占比54.71\%,差评占比14.96\%, 观众主要分布在北京、广东。
正面评价主要关键词包括福尔摩斯、女性、推理,负面评价主要关键词包括黑人、莫里亚蒂等。
综上看来,作为一部福尔摩斯电影,喜欢福尔摩斯和悬疑、推理的人群可以一看,但是出于政治正确添加的
角色设定一定程度上影响了本片的评价

总体而言,对于福尔摩斯爱好者值的一看。

\subsection*{珀尔}
\subsubsection*{图表分布}
\begin{figure}[H]
    \centering
    \begin{minipage}[t]{0.48\textwidth}
    \centering
    \includegraphics[width=6cm]{./images/addr9.jpg}
    \caption{观众评论地址分布}
    \end{minipage}
    \begin{minipage}[t]{0.48\textwidth}
    \centering
    \includegraphics[width=6cm]{./images/rating9.jpg}
    \caption{观众评论分数分布}
    \end{minipage}
\end{figure}
\begin{figure}[H]
    \centering
    \includegraphics[width=10cm]{./images/comments9.jpg}
    \caption{影片评论情感分数分布} 
\end{figure} 
\subsubsection*{Keywords}
正面评论和负面评论关键词
\paragraph*{pos\_word:}
女主:66,电影:65,独白:58,最后:58,演技:54,米娅:52,高斯:50,恐怖片:45,真的:42,恐怖:40,表演:39,压抑:37,复古:34,人物:34,角色:33,故事:32,感觉:30,长镜头:30,这部:29,家庭:28,结尾:28,镜头:28,珀尔:27,母亲:26,农场:25,血腥:24,导演:23,主演:23,一部:23,这种:22
\paragraph*{neg\_word:}
演技:12,最后:10,恐怖:9,电影:8,真的:7,好看:6,高斯:5,女主:5,可能:5,变态:5,一点:4,太好了:4,知道:4,疯批:4,疫情:4,一部:4,不要:4,故事:4,有点:4,镜头:3,危笑:3,剧情:3,神经病:3,喜欢:3,这部:3,已经:3,恐怖片:3,家庭:3,完全:3,开关:3
\subsubsection*{综合评价}
综合评价值得一看, 好评率52.27\%,中评占比38.22\%,差评占比9.50\%, 
观众主要分布于北京、广东。
正面评论关键词为女主,独白,最后,演技,负面评论主要关键词为演技,恐怖,变态等。
作为恐怖片,好评主要集中在女主最后的独白的演技上,相比之下恐怖的成分却要逊色很多。

总体而言,评价较为可观,值得一看。


\subsection*{新神榜:杨戬}
\subsubsection*{图表分布}
\begin{figure}[H]
    \centering
    \begin{minipage}[t]{0.48\textwidth}
    \centering
    \includegraphics[width=6cm]{./images/addr10.jpg}
    \caption{观众评论地址分布}
    \end{minipage}
    \begin{minipage}[t]{0.48\textwidth}
    \centering
    \includegraphics[width=6cm]{./images/rating10.jpg}
    \caption{观众评论分数分布}
    \end{minipage}
\end{figure}
\begin{figure}[H]
    \centering
    \includegraphics[width=10cm]{./images/comments10.jpg}
    \caption{影片评论情感分数分布} 
\end{figure} 
\subsubsection*{Keywords}
正面评论和负面评论关键词
\paragraph*{pos\_word:}
杨戬:226,故事:154,剧情:143,追光:113,人物:101,电影:88,画面:86,真的:86,沉香:86,动画:79,特效:75,朋克:61,角色:59,劈山:56,编剧:54,美术:51,哪吒:50,最后:50,设定:48,申公豹:45,剧本:45,觉得:44,感觉:44,中国:43,场景:43,有点:39,蒸汽:38,救母:38,不能:37,神话:37
\paragraph*{neg\_word:}
剧情:28,特效:23,编剧:22,故事:18,真的:18,杨戬:17,追光:17,画面:12,沉香:11,不能:11,美术:10,有点:9,感觉:8,觉得:7,最后:7,无聊:7,剧本:7,知道:6,人物:6,劈山:6,好看:6,求求:5,希望:5,东西:5,恶心:5,朋克:5,逻辑:5,浪费:5,两个:5,不好:5
\subsubsection*{综合评价}
整体评价中等偏上,好评占比43.46\%, 中评占比33.00\%, 差评占比23.54\%。观众主要分布在
北京、上海、广东,正面评论主要关键词为杨戬、故事、剧情、追光、人物,负面评论主要关键词包括剧情
特效。
作为一部动画电影,人设到位,主人公杨戬极为突出,特效和美工都受到了欢迎。但是剧情相对有所欠缺。

总体而言,忽略一些剧情上的缺点,值得一看。

\subsection*{总结}
综上所述,上述电影可分为以下几个类别:

非常推荐:西线无战事,火山挚恋
比较推荐或特定人群可看:名侦探柯南:万圣节的新娘,6/48,共助2:国际,福尔摩斯小姐:伦敦厄运,新神榜:杨戬
不推荐观看:危笑,黑亚当

同时,从影片粉丝地域分布可以看出,国内电影业受疫情影响较大,除了几大超一线城市外,其他城市
人群的观影热情相对而言受到了较大影响。

\section{Future Work}
本文仍然存在以下缺陷: 
\begin{itemize}
    \item 由于豆瓣对爬虫的限制,每个电影只能爬取到600条左右的评论数据,数据规模不是很大。
    \item 由于缺乏数据集,使用的是预训练好的模型,缺乏对于电影评论的针对性
    \item 对于一段评论,可能同时存在多个正负面不同的观点,对一段评论用统一的一个情感因子评价,难以确切的衡量对于评论中某些特定关键词的态度,因此,对于关键词的分析,还不够精确。
\end{itemize}

所以,在时间充足的情况下,我会考虑如下优化:
\begin{itemize}
    \item 广泛收集多个网站的数据,自己进行模型的训练
    \item 对于一段评论的情感分析,考虑上下文的建模,对评论进行细粒度的分析,使得提取出的负面和正面关键词能更具有代表性。\cite{pottsDynaSentDynamicBenchmark2020}
\end{itemize}

\section{Referance}
\bibliography{./cite/Dyna} 

\end{document}